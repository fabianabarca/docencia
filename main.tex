\documentclass{memoir}

\usepackage[spanish]{babel}
\usepackage{libertine}
\usepackage{libertinust1math}
\usepackage[T1]{fontenc}
\usepackage{lipsum}

\usepackage{markdown}

\title{Estrategias docentes para sesiones virtuales interactivas con el desarrollo de un nuevo sistema web: una experiencia en el curso Modelos Probabilísticos de Señales y Sistemas}
\author{Fabián Abarca Calderón}
\date{Julio de 2025}

\begin{document}

\maketitle
\newpage
\begin{abstract}
    \markdownInput{docs/0_resumen.md}
\end{abstract}
\newpage
\tableofcontents

%%%%%%%%%%%%%%%%%
\part{Desarrollo}
%%%%%%%%%%%%%%%%%

\chapter*{Introducción}

\markdownInput{docs/1_0_introduccion.md}

% --------------------
\chapter{Antecedentes}
% --------------------

\markdownInput{docs/1_1_antecedentes.md}

% ---------------------
\chapter{Justificación}
% ---------------------

\markdownInput{docs/1_2_justificacion.md}

%%%%%%%%%%%%%
\part{Kalouk}
%%%%%%%%%%%%%

\chapter*{Introducción}

\markdownInput{docs/2_0_introduccion.md}

% ----------------------------------
\chapter{Desarrollo del sistema web}
% ----------------------------------

\markdownInput{docs/2_1_desarrollo.md}

% --------------------------------
\chapter{Diseño de interactividad}
% --------------------------------

\markdownInput{docs/2_2_interactividad.md}

\end{document}
